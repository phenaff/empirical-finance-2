\documentclass[]{tufte-book}

% ams
\usepackage{amssymb,amsmath}

\usepackage{ifxetex,ifluatex}
\usepackage{fixltx2e} % provides \textsubscript
\ifnum 0\ifxetex 1\fi\ifluatex 1\fi=0 % if pdftex
  \usepackage[T1]{fontenc}
  \usepackage[utf8]{inputenc}
\else % if luatex or xelatex
  \makeatletter
  \@ifpackageloaded{fontspec}{}{\usepackage{fontspec}}
  \makeatother
  \defaultfontfeatures{Ligatures=TeX,Scale=MatchLowercase}
  \makeatletter
  \@ifpackageloaded{soul}{
     \renewcommand\allcapsspacing[1]{{\addfontfeature{LetterSpace=15}#1}}
     \renewcommand\smallcapsspacing[1]{{\addfontfeature{LetterSpace=10}#1}}
   }{}
  \makeatother

\fi

% graphix
\usepackage{graphicx}
\setkeys{Gin}{width=\linewidth,totalheight=\textheight,keepaspectratio}

% booktabs
\usepackage{booktabs}

% url
\usepackage{url}

% hyperref
\usepackage{hyperref}

% units.
\usepackage{units}


\setcounter{secnumdepth}{2}

% citations


% pandoc syntax highlighting
\usepackage{color}
\usepackage{fancyvrb}
\newcommand{\VerbBar}{|}
\newcommand{\VERB}{\Verb[commandchars=\\\{\}]}
\DefineVerbatimEnvironment{Highlighting}{Verbatim}{commandchars=\\\{\}}
% Add ',fontsize=\small' for more characters per line
\newenvironment{Shaded}{}{}
\newcommand{\AlertTok}[1]{\textcolor[rgb]{1.00,0.00,0.00}{\textbf{#1}}}
\newcommand{\AnnotationTok}[1]{\textcolor[rgb]{0.38,0.63,0.69}{\textbf{\textit{#1}}}}
\newcommand{\AttributeTok}[1]{\textcolor[rgb]{0.49,0.56,0.16}{#1}}
\newcommand{\BaseNTok}[1]{\textcolor[rgb]{0.25,0.63,0.44}{#1}}
\newcommand{\BuiltInTok}[1]{\textcolor[rgb]{0.00,0.50,0.00}{#1}}
\newcommand{\CharTok}[1]{\textcolor[rgb]{0.25,0.44,0.63}{#1}}
\newcommand{\CommentTok}[1]{\textcolor[rgb]{0.38,0.63,0.69}{\textit{#1}}}
\newcommand{\CommentVarTok}[1]{\textcolor[rgb]{0.38,0.63,0.69}{\textbf{\textit{#1}}}}
\newcommand{\ConstantTok}[1]{\textcolor[rgb]{0.53,0.00,0.00}{#1}}
\newcommand{\ControlFlowTok}[1]{\textcolor[rgb]{0.00,0.44,0.13}{\textbf{#1}}}
\newcommand{\DataTypeTok}[1]{\textcolor[rgb]{0.56,0.13,0.00}{#1}}
\newcommand{\DecValTok}[1]{\textcolor[rgb]{0.25,0.63,0.44}{#1}}
\newcommand{\DocumentationTok}[1]{\textcolor[rgb]{0.73,0.13,0.13}{\textit{#1}}}
\newcommand{\ErrorTok}[1]{\textcolor[rgb]{1.00,0.00,0.00}{\textbf{#1}}}
\newcommand{\ExtensionTok}[1]{#1}
\newcommand{\FloatTok}[1]{\textcolor[rgb]{0.25,0.63,0.44}{#1}}
\newcommand{\FunctionTok}[1]{\textcolor[rgb]{0.02,0.16,0.49}{#1}}
\newcommand{\ImportTok}[1]{\textcolor[rgb]{0.00,0.50,0.00}{\textbf{#1}}}
\newcommand{\InformationTok}[1]{\textcolor[rgb]{0.38,0.63,0.69}{\textbf{\textit{#1}}}}
\newcommand{\KeywordTok}[1]{\textcolor[rgb]{0.00,0.44,0.13}{\textbf{#1}}}
\newcommand{\NormalTok}[1]{#1}
\newcommand{\OperatorTok}[1]{\textcolor[rgb]{0.40,0.40,0.40}{#1}}
\newcommand{\OtherTok}[1]{\textcolor[rgb]{0.00,0.44,0.13}{#1}}
\newcommand{\PreprocessorTok}[1]{\textcolor[rgb]{0.74,0.48,0.00}{#1}}
\newcommand{\RegionMarkerTok}[1]{#1}
\newcommand{\SpecialCharTok}[1]{\textcolor[rgb]{0.25,0.44,0.63}{#1}}
\newcommand{\SpecialStringTok}[1]{\textcolor[rgb]{0.73,0.40,0.53}{#1}}
\newcommand{\StringTok}[1]{\textcolor[rgb]{0.25,0.44,0.63}{#1}}
\newcommand{\VariableTok}[1]{\textcolor[rgb]{0.10,0.09,0.49}{#1}}
\newcommand{\VerbatimStringTok}[1]{\textcolor[rgb]{0.25,0.44,0.63}{#1}}
\newcommand{\WarningTok}[1]{\textcolor[rgb]{0.38,0.63,0.69}{\textbf{\textit{#1}}}}

% table with pandoc
\usepackage{longtable,booktabs,array}
\usepackage{calc} % for calculating minipage widths
% Correct order of tables after \paragraph or \subparagraph
\usepackage{etoolbox}
\makeatletter
\patchcmd\longtable{\par}{\if@noskipsec\mbox{}\fi\par}{}{}
\makeatother
% Allow footnotes in longtable head/foot
\IfFileExists{footnotehyper.sty}{\usepackage{footnotehyper}}{\usepackage{footnote}}
\makesavenoteenv{longtable}

% multiplecol
\usepackage{multicol}

% strikeout
\usepackage[normalem]{ulem}

% morefloats
\usepackage{morefloats}


% tightlist macro required by pandoc >= 1.14
\providecommand{\tightlist}{%
  \setlength{\itemsep}{0pt}\setlength{\parskip}{0pt}}

% title / author / date
\date{}

% Preamble from Rmetrics

\usepackage{booktabs}
\usepackage{amsthm}
\usepackage{xfrac}
\makeatletter
\def\thm@space@setup{%
  \thm@preskip=8pt plus 2pt minus 4pt
  \thm@postskip=\thm@preskip
}
\makeatother

% Index
\usepackage{makeidx}
\makeindex

% binomial trees
\usepackage{pgfplots}
\usepackage{tikz}
\usetikzlibrary{shapes.multipart}
\usepackage{pgfplots}
\usetikzlibrary{shapes}
\usetikzlibrary{external}
\usepgfplotslibrary{external}
\usetikzlibrary{positioning}
\pgfplotsset{compat=1.18}

% R and other languages

\newcommand{\RR}{\textsf{R}}
\newcommand{\Rmetrics}{Rmetrics}

%\newcommand{`r to_index("", "functions")`}[1]{\texttt{#1()}\index{R~functions@\RR~functions!#1}}
%\newcommand{`r to_index("", "classes")`}[1]{\texttt{#1}}
%\newcommand{`r to_index("", "classes")`}[1]{\texttt{#1}\index{R~classes@\RR~classes!#1}}
%\newcommand{\pkg}[1]{\texttt{#1}\index{R~packages@\RR~packages!#1}}
%\newcommand{\dataset}[1]{\texttt{#1}\index{R~data@\RR~data!#1}}
%\newcommand{\code}[1]{\texttt{#1}\index{#1}}

% Preamble from VIP

\newcommand{\given}{\mid}
\renewcommand{\neg}{\mathbin{\sim}}
\renewcommand{\wedge}{\mathbin{\&}}
\renewcommand{\u}{U}
\newcommand{\gt}{>}
\newcommand{\p}{Pr}
\newcommand{\E}{E}
\newcommand{\EU}{EU}
\newcommand{\pr}{Pr}
\newcommand{\po}{Pr^*}
\newcommand{\degr}{^{\circ}}
\definecolor{bookred}{RGB}{228,6,19}
\definecolor{bookblue}{RGB}{0,92,169}
\definecolor{bookpurple}{RGB}{114,49,94}

\newenvironment{epigraph}%
{
\begin{flushright}
\begin{minipage}{20em}
\begin{flushright}
\itshape
}%
{
\end{flushright}
\end{minipage}
\end{flushright}
}
\newenvironment{problem}{\begin{quote}\normalsize}{\end{quote}}
\newenvironment{puzzle}{\begin{quote}\normalsize}{\end{quote}}
\def\argument{\list{}{\leftmargin3em}\item[]}
\let\endargument=\endlist
\usepackage{fontawesome}
\newenvironment{warning}{\begin{itemize}\item[\faBan]}{\end{itemize}}
\usepackage{marvosym}
\newenvironment{info}{\begin{itemize}\item[\Info]}{\end{itemize}}

%%%% Kevin Godny's code for title page and contents from https://groups.google.com/forum/#!topic/tufte-latex/ujdzrktC1BQ
\makeatletter
\renewcommand{\maketitlepage}{%
\begingroup%
\setlength{\parindent}{0pt}
{\fontsize{18}{18}\selectfont\textit{\@author}\par}
\vspace{1.75in}{\fontsize{36}{14}\selectfont\@title\par}
\vspace{0.5in}{\fontsize{20}{14}\selectfont with R and Rmetrics\par}
\vspace{0.5in}{\fontsize{14}{14}\selectfont\textsf{\smallcaps{v 2.0}}\par}
\vfill{\fontsize{14}{14}\selectfont\textit{An Open Access Publication}\par}
\thispagestyle{empty}
\endgroup
}
\makeatother

% Change shape from [display] to [block] to keep chapter numbers and titles on the same line
\titleformat{\chapter}%
  [block]% shape
  {\relax\ifthenelse{\NOT\boolean{@tufte@symmetric}}{\begin{fullwidth}}{}}% format applied to label+text
  {\itshape\huge\thechapter}% label
  {3em}% horizontal separation between label and title body
  {\huge\rmfamily\itshape}% before the title body
  [\ifthenelse{\NOT\boolean{@tufte@symmetric}}{\end{fullwidth}}{}]% after the title body


\usepackage{etoolbox}
% Jesse Rosenthal's code from https://groups.google.com/forum/#!topic/pandoc-discuss/wCF78X6SvwY
% Avoid new pagraph/indent after lists, quotes, etc.
\makeatletter
\newcommand{\gobblepars}{%
    \@ifnextchar\par%
        {\expandafter\gobblepars\@gobble}%
        {}}
\newcommand{\eatpar}{\@ifnextchar\par{\@gobble}{}}
\newcommand{\forcepar}{\par}
\makeatother
\AfterEndEnvironment{quote}{\expandafter\gobblepars}
\AfterEndEnvironment{enumerate}{\expandafter\gobblepars}
\AfterEndEnvironment{itemize}{\expandafter\gobblepars}
\AfterEndEnvironment{description}{\expandafter\gobblepars}
\AfterEndEnvironment{example}{\expandafter\gobblepars}
\AfterEndEnvironment{argument}{\expandafter\gobblepars}
\AfterEndEnvironment{problem}{\expandafter\gobblepars}
\AfterEndEnvironment{info}{\expandafter\gobblepars}
\AfterEndEnvironment{warning}{\expandafter\gobblepars}
\AfterEndEnvironment{marginfigure}{\expandafter\gobblepars}
\AfterEndEnvironment{longtable}{\expandafter\gobblepars} % not working, why?
\makeatletter
\AfterEndEnvironment{longtable}{\par\@afterindentfalse\@afterheading} % this seems to work instead
\makeatother

\renewcommand*\descriptionlabel[1]{\hspace\labelsep\normalfont\em #1.}

% prevent extra space when \newthought follows \section
% see: https://tex.stackexchange.com/questions/291746/tufte-latex-newthought-after-section
\makeatletter
\def\tuftebreak{%
  \if@nobreak\else
    \par
    \ifdim\lastskip<\tufteskipamount
      \removelastskip \penalty -100
      \tufteskip
    \fi
  \fi
}
\makeatother

% indent lists a bit
\usepackage{enumitem}
\setlist[1]{leftmargin=24pt}

\def\labelitemii{$\circ$}

\begin{document}




{
\setcounter{tocdepth}{0}
\tableofcontents
}

\hypertarget{vanna-volga-pricing-and-hedging}{%
\chapter{Vanna-Volga Pricing and Hedging}\label{vanna-volga-pricing-and-hedging}}

{\emph{If you want to know the value of a security, use the price of another\\
security that's similar to it. All the rest is strategy.}}\\
E. Derman

This chapter presents a practical method for pricing and hedging
derivatives, taking into account an uncertain volatility. This method is
very popular for Foreign Exchange derivatives, but beyond that it
illustrates an important principle of asset pricing, which is to relate
the price of a complex derivative to the known price of simpler, liquid
instruments.

\hypertarget{principle}{%
\section{Principle}\label{principle}}

The goal is to value an arbitrary option contract \(O\) by constructing a hedged
portfolio that is delta-neutral and vega-neutral in the Black-Scholes world (that is, in a world with a flat smile). We assume that the option price can be modeled by the Black-Scholes PDE with a flat but stochastic volatility. The hedged portfolio \(\Pi\), having a long position in \(O\), includes 4 additional assets: a short position of \(\Delta_t\) units of the underlying asset, and another short position of \(x_i\) units of three benchmark vanilla options.

\[
\Pi(t) = O(t) - \Delta_t S_t - \sum_{i=1}^3 x_i C_i(t)
\]
We now show that when volatility is flat but stochastic, and the options are valued with the Black-Scholes formula, we can still have a dynamic perfect hedge, with a proper choice of the replicating portfolio \(x_i, i=1, \ldots, 3\) to cancel model risk.

The change in portfolio value over a small time interval \(dt\) is:
\[
d\Pi(t) = dO(t) - \Delta_t dS_t -\sum_{i=1}^3 x_i dC_i(t)
\]

An extended version of Ito's lemma, were terms in \(dS_td_t, dtd\sigma_t\) and \((dt)^2\) vanish, is given below:

\begin{align}
  dO(t) &= \frac{\partial O}{\partial t} dt + \frac{\partial O}{\partial S} dS_t + \frac{\partial O}{\partial \sigma} d\sigma_t \\
  &  + \frac{\partial^2 O}{\partial S^2} (dS_t)^2 + \frac{\partial^2 O}{\partial \sigma^2} (d\sigma_t)^2 + \frac{\partial^2 O}{\partial S \partial \sigma} dS_t d\sigma_t
  \end{align}

The corresponding change in portfolio \(\Pi\) is

\begin{align}
  d\Pi & = \left[ \frac{\partial O}{\partial t} - \sum_i x_i \frac{\partial C_i}{\partial t}  \right] dt  \\
  & + \left[ \frac{\partial O}{\partial S} - \Delta_t - \sum_i x_i \frac{\partial C_i}{\partial S}  \right] dS_t \\
  & + \left[ \frac{\partial O}{\partial \sigma} - \sum_i x_i \frac{\partial C_i}{\partial \sigma}  \right] d\sigma_t \\
  & + \frac{1}{2} \left[ \frac{\partial^2 O}{\partial S^2} - \sum_i x_i \frac{\partial^2 C_i}{\partial S^2}  \right] (dS_t)^2 \\
  & + \frac{1}{2} \left[ \frac{\partial^2 O}{\partial \sigma^2} - \sum_i x_i \frac{\partial^2 C_i}{\partial \sigma^2}  \right] (d\sigma_t)^2  \\
  & + \left[ \frac{\partial^2 O}{\partial S \partial \sigma} - \sum_i x_i \frac{\partial^2 C_i}{\partial S \partial \sigma}  \right] dS_t d\sigma_t
 \end{align}

We calculate \((dS_t)^2\), retaining the first order terms:

\begin{align}
(dS_t)^2 &= (rS_tdt + \sigma S_tdW_t) (rS_tdt + \sigma S_tdW_t) \\
&= \sigma^2S_t^2(dW)^2 \\
&= \sigma_t^2 S_t^2 dt
\end{align}

We also choose \(\Delta_t\) and \(x_i\) to zero out the terms
\(dS_t, d\sigma_t, (d\sigma_t)^2\) and \(dS_t d\sigma_t\). We are left with:

\[
d\Pi = \left[\left( \frac{\partial O}{\partial t} - \sum_{i=1}^3 x_i \frac{\partial C_i}{\partial t} \right) + \frac{1}{2} \sigma^2_t S^2_t \left(\frac{\partial^2 O}{\partial S^2_t} - 3\sum_{i=1}^3 x_i \frac{\partial^2 C_i}{\partial S_i^2} \right) \right] dt 
\]

The hedged portfolio is thus riskless, and must earn the riskless rate:

\begin{equation}
  d\Pi =  r \left[ O(t) - -\Delta_t S_t - \sum_i x_i C_i(t) \right] dt
\label{eq:vv-1}
\end{equation}

In summary, we can still have a locally perfect hedge when volatility is
stochastic, as long as the prices \(O(t)\) and \(C_i(t)\) follow the
Black-Scholes equation.

\hypertarget{vanna-volga-option-pricing}{%
\section{Vanna-Volga option pricing}\label{vanna-volga-option-pricing}}

Up to now, we have assumed that option \(O\) as well as \(C_i\) were all priced according to the Black-Scholes model. In reality, the benchmark options have a quoted market price \(C_i^M(t)\) which is different from the flat volatility Black-Scholes price \(C_i^{BS}(t)\). We need to determine the price \(O^M(t)\) which is consistent with the market prices of the benchmark options. An approximate argument is presented below; see @Shkolnikov
for a rigorous treatment.

Discretize equation~\eqref{eq:vv-1} at time \(t=T-\delta t\), where \(T\) is the
option expiry, noting that, at expiry, the market prices and Black-Scholes prices of the options are identical. We get:

\begin{align}
O(T) - O(t) - \Delta_t (S_T - S_t) - \sum_i x_i \left[C_i(T)-C_i^{BS}(t) \right] = \\
r \left[O(t) - \sum_i x_i C^{BS}(t) - \Delta_t S_t \right] \delta t
\label{eq:vv-2}
\end{align}

By setting
\[
O(t) = O^{BS}(t) + \sum_i x_i \left[ C^M_i(t,K_i) - C^{BS}_i(t,K_i) \right]
\label{eq:vv-3}
\]
and substituting in \eqref{eq:vv-2}, we get
\begin{align}
O(T) &= O(t) + \Delta_t (S_T - S_0) \\
&+ \sum_i x_i \left[C_i(T) - C^M_i(t) \right] \\
&+ r \left(O(t) - \sum_i x_i C^M_i(t) - \Delta S_t \right) \delta t
\end{align}

In summary, if we have a wealth \(O(t)\) defined by~\eqref{eq:vv-3} at
time \(T-\delta t\), then we can replicate the payoff \(O(T)\) at expiry,
with a hedge at market price. The argument made on the interval
\([T-\delta t, T]\) can be applied by backward recursion for each time
interval until \(t=0\).

We have both a hedging strategy and a process for adjusting the price of
any derivative to account for the smile. let's now consider some
implementation details.

\hypertarget{implementation}{%
\section{Implementation}\label{implementation}}

The weights \(x_i\) are obtained by solving the system of linear
equations:

\begin{align}
 \frac{\partial O}{\partial \sigma} &= \sum_i x_i \frac{\partial C_i}{\partial \sigma} \\
\frac{\partial^2 O}{\partial \sigma^2} &= \sum_i x_i \frac{\partial^2 C_i}{\partial \sigma^2} \\
\frac{\partial^2 O}{\partial S \partial \sigma} &= \sum_i x_i \frac{\partial^2 C_i}{\partial S \partial \sigma}
\end{align}

or,
\begin{equation}
b = Ax
\label{eq:A-matrix}
\end{equation}

Since the result of the previous section holds for any derivative that
verifies the Black-Scholes equation, we can choose the benchmark
securities \(C_i\) as we see fit.

To simplify notation, we denote \(C(K), P(K)\) the call
and put of strike \(K\), maturity \(T\).
A popular set of benchmark securities, commonly used in the FX market, is
in fact a set of benchmark portfolios:

\begin{itemize}
\item
  An at-the-money straddle: \[C_1 = C(S) + P(S)\]
\item
  A ``risk reversal'', traditionally defined as
  \[C_2 = P(K_1) - C(K_2)\] with \(K_1\) and \(K_2\) chosen so that the
  options have a Delta of .25 in absolute value.
\item
  A ``butterfly'', defined as
  \[C_3 = \beta ( P(K_1) + C(K_2) ) -(P(S)+C(S))\] with \(\beta\)
  determined to set the Vega of the butterfly to 0.
\end{itemize}

This system is popular because the benchmark securities are very liquid,
and because the resulting \(A\) matrix of \eqref{eq:A-matrix} is almost
diagonal, which allows an intuitive interpretation of the coefficients
\(x_i\).

To summarize, the calculation steps for pricing an option, taking the
smile cost into account, are as follows:

\begin{enumerate}
\def\labelenumi{\arabic{enumi}.}
\item
  compute the risk indicators for the option \(O\) to be priced:
  \[b =  \begin{pmatrix}
            \frac{\partial O}{\partial \sigma} \\
            \frac{\partial^2 O}{\partial \sigma^2} \\
            \frac{\partial^2 O}{\partial \sigma \partial S}
          \end{pmatrix}\]
\item
  compute the A matrix \[A =
        \begin{pmatrix}
            \frac{\partial C_1}{\partial \sigma} & \ldots & \frac{\partial C_3}{\partial \sigma}\\
            \frac{\partial^2 C_1}{\partial \sigma^2} & \ldots & \frac{\partial^2 C_3}{\partial \sigma^2}  \\
            \frac{\partial^2 C_1}{\partial \sigma \partial S}
   & \ldots & \frac{\partial^2 C_3}{\partial \sigma \partial S}
  \end{pmatrix}\]
\item
  solve for \(x\): \[b = Ax\]
\item
  the corrected price for \(O\) is:
  \[O^M(t,K) = O^{BS}(t,K) + \sum_{i=2}^3 x_i \left( C_i^M(t) - C_i^{BS}(t) \right)
          \label{eq:vv-5}\]
  where \(C^M_i(t)\) is the market price and
  \(C^{BS}_i(t)\) the Black-Scholes price (i.e.~with flat volatility).
\end{enumerate}

The term in \(x_1\) is omitted in \eqref{eq:vv-5} since, by definition,
the Black-Scholes price and market price of an ATM straddle are
identical.

Neglecting the off diagonal terms in \(A\), a simplified procedure is to
estimate \(x_i\) by:
\begin{align}
  x_2 &=& \frac{\frac{\partial^2 O}{\partial \sigma^2}}{\frac{\partial^2 C_2}{\partial \sigma^2}} \\
  x_3 &=&  \frac{\frac{\partial^2 O}{\partial \sigma \partial S}}{\frac{\partial^2 C_3}{\partial \sigma \partial S}}
\end{align}

\hypertarget{volatility-interpolation}{%
\subsection{Volatility Interpolation}\label{volatility-interpolation}}

The simplest use of this method is volatility interpolation. Given the ATM
volatility and at two other strikes, we want to determine the volatility
at an arbitrary strike \(K\).

The process is illustrated below. The volatility of the three benchmark
instruments is provided in
Table~\ref{tab:vv-10} for European options with maturity \(T=1\) year.
Interest rate is set to 0 for simplicity.

\label{tab:vv-10}Volatility of benchmark instruments

Strike

Vol

80

0.320

100

0.300

120

0.315

\begin{Shaded}
\begin{Highlighting}[]
\NormalTok{  T }\OtherTok{\textless{}{-}} \DecValTok{1}
\NormalTok{  Spot }\OtherTok{\textless{}{-}} \DecValTok{100}
\NormalTok{  r }\OtherTok{\textless{}{-}} \DecValTok{0}
\NormalTok{  b }\OtherTok{\textless{}{-}} \DecValTok{0}
\NormalTok{  eps }\OtherTok{\textless{}{-}}\NormalTok{ .}\DecValTok{001}
\NormalTok{  sigma }\OtherTok{\textless{}{-}}\NormalTok{ .}\DecValTok{3}

  \CommentTok{\# Benchmark data: (strike, volatility)}
\NormalTok{  VolData }\OtherTok{\textless{}{-}} \FunctionTok{list}\NormalTok{(}\FunctionTok{c}\NormalTok{(}\DecValTok{80}\NormalTok{, .}\DecValTok{32}\NormalTok{), }\FunctionTok{c}\NormalTok{(}\DecValTok{100}\NormalTok{, .}\DecValTok{30}\NormalTok{), }\FunctionTok{c}\NormalTok{(}\DecValTok{120}\NormalTok{, .}\DecValTok{315}\NormalTok{))}
\end{Highlighting}
\end{Shaded}

Define an array of pricing functions for the three benchmark
instruments:

\begin{Shaded}
\begin{Highlighting}[]
\NormalTok{C }\OtherTok{\textless{}{-}} \FunctionTok{c}\NormalTok{(}\ControlFlowTok{function}\NormalTok{(}\AttributeTok{vol =}\NormalTok{ sigma, }\AttributeTok{spot =}\NormalTok{ Spot) }\FunctionTok{GBSOption}\NormalTok{(}\AttributeTok{TypeFlag =} \StringTok{"c"}\NormalTok{,}
    \AttributeTok{S =}\NormalTok{ spot, }\AttributeTok{X =}\NormalTok{ VolData[[}\DecValTok{1}\NormalTok{]][}\DecValTok{1}\NormalTok{], }\AttributeTok{Time =}\NormalTok{ T, }\AttributeTok{r =}\NormalTok{ r, }\AttributeTok{b =}\NormalTok{ b, }\AttributeTok{sigma =}\NormalTok{ vol)}\SpecialCharTok{@}\NormalTok{price,}
    \ControlFlowTok{function}\NormalTok{(}\AttributeTok{vol =}\NormalTok{ sigma, }\AttributeTok{spot =}\NormalTok{ Spot) }\FunctionTok{GBSOption}\NormalTok{(}\AttributeTok{TypeFlag =} \StringTok{"c"}\NormalTok{,}
        \AttributeTok{S =}\NormalTok{ spot, }\AttributeTok{X =}\NormalTok{ VolData[[}\DecValTok{2}\NormalTok{]][}\DecValTok{1}\NormalTok{], }\AttributeTok{Time =}\NormalTok{ T, }\AttributeTok{r =}\NormalTok{ r, }\AttributeTok{b =}\NormalTok{ b,}
        \AttributeTok{sigma =}\NormalTok{ vol)}\SpecialCharTok{@}\NormalTok{price, }\ControlFlowTok{function}\NormalTok{(}\AttributeTok{vol =}\NormalTok{ sigma, }\AttributeTok{spot =}\NormalTok{ Spot) }\FunctionTok{GBSOption}\NormalTok{(}\AttributeTok{TypeFlag =} \StringTok{"c"}\NormalTok{,}
        \AttributeTok{S =}\NormalTok{ spot, }\AttributeTok{X =}\NormalTok{ VolData[[}\DecValTok{3}\NormalTok{]][}\DecValTok{1}\NormalTok{], }\AttributeTok{Time =}\NormalTok{ T, }\AttributeTok{r =}\NormalTok{ r, }\AttributeTok{b =}\NormalTok{ b,}
        \AttributeTok{sigma =}\NormalTok{ vol)}\SpecialCharTok{@}\NormalTok{price)}
\end{Highlighting}
\end{Shaded}

Next, define utility functions to compute the risk indicators, all by
finite difference:

\begin{Shaded}
\begin{Highlighting}[]
\NormalTok{  Vega }\OtherTok{\textless{}{-}} \ControlFlowTok{function}\NormalTok{(f, vol, }\AttributeTok{spot=}\NormalTok{Spot) (}\FunctionTok{f}\NormalTok{(vol}\SpecialCharTok{+}\NormalTok{eps, spot)}\SpecialCharTok{{-}}\FunctionTok{f}\NormalTok{(vol}\SpecialCharTok{{-}}\NormalTok{eps, spot))}\SpecialCharTok{/}\NormalTok{(}\DecValTok{2}\SpecialCharTok{*}\NormalTok{eps)}

\NormalTok{  Vanna }\OtherTok{\textless{}{-}} \ControlFlowTok{function}\NormalTok{(f, vol, }\AttributeTok{spot=}\NormalTok{Spot) \{}
\NormalTok{    (}\FunctionTok{Vega}\NormalTok{(f, vol, spot}\SpecialCharTok{+}\DecValTok{1}\NormalTok{)}\SpecialCharTok{{-}}\FunctionTok{Vega}\NormalTok{(f, vol, spot}\DecValTok{{-}1}\NormalTok{))}\SpecialCharTok{/}\DecValTok{2}
\NormalTok{  \}}

\NormalTok{  Volga }\OtherTok{\textless{}{-}} \ControlFlowTok{function}\NormalTok{(f, vol) \{}
\NormalTok{      (}\FunctionTok{Vega}\NormalTok{(f,vol}\SpecialCharTok{+}\NormalTok{eps)}\SpecialCharTok{{-}}\FunctionTok{Vega}\NormalTok{(f,vol}\SpecialCharTok{{-}}\NormalTok{eps))}\SpecialCharTok{/}\NormalTok{(eps)}
\NormalTok{  \}}
\end{Highlighting}
\end{Shaded}

Finally, the following function computes the Vanna-Volga adjustment to
the Black-Scholes price, and the corresponding implied volatility:

\begin{Shaded}
\begin{Highlighting}[]
\NormalTok{  VVVol }\OtherTok{\textless{}{-}} \ControlFlowTok{function}\NormalTok{(X) \{}

\NormalTok{  O }\OtherTok{\textless{}{-}} \ControlFlowTok{function}\NormalTok{(}\AttributeTok{vol=}\NormalTok{sigma, }\AttributeTok{spot=}\NormalTok{Spot) }\FunctionTok{GBSOption}\NormalTok{(}\AttributeTok{TypeFlag=}\StringTok{\textquotesingle{}c\textquotesingle{}}\NormalTok{, }\AttributeTok{S=}\NormalTok{spot,}
         \AttributeTok{X=}\NormalTok{X, }\AttributeTok{Time=}\NormalTok{T, }\AttributeTok{r=}\NormalTok{r, }\AttributeTok{b=}\NormalTok{b, }\AttributeTok{sigma=}\NormalTok{vol)}\SpecialCharTok{@}\NormalTok{price}
\NormalTok{  TV.BS }\OtherTok{\textless{}{-}} \FunctionTok{O}\NormalTok{()}

  \CommentTok{\# risk indicators for benchmark instruments}
\NormalTok{  B.vega }\OtherTok{\textless{}{-}} \FunctionTok{sapply}\NormalTok{(}\DecValTok{1}\SpecialCharTok{:}\DecValTok{3}\NormalTok{, }\ControlFlowTok{function}\NormalTok{(i) }\FunctionTok{Vega}\NormalTok{(C[[i]], sigma))}
\NormalTok{  B.vanna }\OtherTok{\textless{}{-}} \FunctionTok{sapply}\NormalTok{(}\DecValTok{1}\SpecialCharTok{:}\DecValTok{3}\NormalTok{, }\ControlFlowTok{function}\NormalTok{(i) }\FunctionTok{Vanna}\NormalTok{(C[[i]], sigma))}
\NormalTok{  B.volga }\OtherTok{\textless{}{-}} \FunctionTok{sapply}\NormalTok{(}\DecValTok{1}\SpecialCharTok{:}\DecValTok{3}\NormalTok{, }\ControlFlowTok{function}\NormalTok{(i) }\FunctionTok{Volga}\NormalTok{(C[[i]], sigma))}

  \CommentTok{\# risk indicators for new option}
\NormalTok{  O.vega }\OtherTok{\textless{}{-}} \FunctionTok{Vega}\NormalTok{(O, sigma)}
\NormalTok{  O.vanna }\OtherTok{\textless{}{-}} \FunctionTok{Vanna}\NormalTok{(O, sigma)}
\NormalTok{  O.volga }\OtherTok{\textless{}{-}} \FunctionTok{Volga}\NormalTok{(O, sigma)}

  \CommentTok{\# Benchmark costs}
\NormalTok{  B.cost }\OtherTok{\textless{}{-}} \FunctionTok{sapply}\NormalTok{(}\DecValTok{1}\SpecialCharTok{:}\DecValTok{3}\NormalTok{, }\ControlFlowTok{function}\NormalTok{(i) C[[i]](VolData[[i]][}\DecValTok{2}\NormalTok{]) }\SpecialCharTok{{-}}\NormalTok{ C[[i]](sigma))}

  \CommentTok{\# calculation of price adjustment}
\NormalTok{  A }\OtherTok{\textless{}{-}} \FunctionTok{t}\NormalTok{(}\FunctionTok{matrix}\NormalTok{(}\FunctionTok{c}\NormalTok{(B.vega, B.vanna, B.volga),  }\AttributeTok{nrow=}\DecValTok{3}\NormalTok{))}
\NormalTok{  x }\OtherTok{\textless{}{-}} \FunctionTok{matrix}\NormalTok{(}\FunctionTok{c}\NormalTok{(O.vega, O.vanna, O.volga), }\AttributeTok{nrow=}\DecValTok{3}\NormalTok{)}
\NormalTok{  w }\OtherTok{\textless{}{-}} \FunctionTok{solve}\NormalTok{(A, x)}
\NormalTok{  CF }\OtherTok{\textless{}{-}} \FunctionTok{t}\NormalTok{(w) }\SpecialCharTok{\%*\%} \FunctionTok{matrix}\NormalTok{(B.cost, }\AttributeTok{nrow=}\DecValTok{3}\NormalTok{)}

  \CommentTok{\# implied volatility}
\NormalTok{  v }\OtherTok{\textless{}{-}} \FunctionTok{GBSVolatility}\NormalTok{(TV.BS}\SpecialCharTok{+}\NormalTok{CF, }\StringTok{\textquotesingle{}c\textquotesingle{}}\NormalTok{, Spot, X, T, r, b, }\DecValTok{1}\NormalTok{.e}\DecValTok{{-}5}\NormalTok{)}

\NormalTok{  v\}}
\end{Highlighting}
\end{Shaded}

We finally use the vanna-volga interpolating function to construct the
interpolated smile curve.

The result is shown in Figure~\ref{fig:vv-111}.

\begin{marginfigure}
\includegraphics{307-BS-VannaVolga_files/figure-latex/vv-111-1} \caption[Interpolated volatility curve]{Interpolated volatility curve. The three red dots represent the benchmark options.}\label{fig:vv-111}
\end{marginfigure}

\hypertarget{pricing-a-binary-option}{%
\subsection{Pricing a Binary Option}\label{pricing-a-binary-option}}

Consider a one-year binary call, struck at the money. Assume that the
smile is quadratic. Again, we assume a null interest rate for
simplicity.

This time, we use the traditional benchmark instruments of the FX
market: straddle, risk-reversal and butterfly, and compute the price of
the binary option, adjusted for the smile effect.

\begin{Shaded}
\begin{Highlighting}[]
\NormalTok{  T }\OtherTok{\textless{}{-}} \DecValTok{1}
\NormalTok{  Spot }\OtherTok{\textless{}{-}} \DecValTok{100}
\NormalTok{  r }\OtherTok{\textless{}{-}} \DecValTok{0}
\NormalTok{  d }\OtherTok{\textless{}{-}} \DecValTok{0}
\NormalTok{  b }\OtherTok{\textless{}{-}}\NormalTok{ r}\SpecialCharTok{{-}}\NormalTok{d}
\NormalTok{  sigma }\OtherTok{\textless{}{-}} \DecValTok{30}\SpecialCharTok{/}\DecValTok{100}
\NormalTok{  X }\OtherTok{\textless{}{-}} \FloatTok{110.50}

  \CommentTok{\# smile function}
\NormalTok{  smile }\OtherTok{\textless{}{-}} \ControlFlowTok{function}\NormalTok{(X) (}\SpecialCharTok{{-}}\NormalTok{(}\DecValTok{0}\SpecialCharTok{/}\DecValTok{20}\NormalTok{)}\SpecialCharTok{*}\NormalTok{(X}\SpecialCharTok{{-}}\NormalTok{Spot) }\SpecialCharTok{+}\NormalTok{ (}\DecValTok{1}\SpecialCharTok{/}\DecValTok{300}\NormalTok{)}\SpecialCharTok{*}\NormalTok{(X}\SpecialCharTok{{-}}\NormalTok{Spot)}\SpecialCharTok{\^{}}\DecValTok{2}\NormalTok{)}\SpecialCharTok{/}\DecValTok{100}
\end{Highlighting}
\end{Shaded}

The strikes corresponding to a \(25\Delta\) call and put are computed by
inverting the formulae for the Delta of European options. Recall that
for a call, the Delta is given by:

\[\Delta = e^{-dT} N(d_1)\]

The strike corresponding to a \(25\Delta\) call is therefore:

\[K_{25\Delta} = S e^{- \left( \sigma \sqrt{T} N^{-1}(e^{dT}.25) -(r-d+\frac{\sigma^2}{2})T \right)}\]

\begin{Shaded}
\begin{Highlighting}[]
  \CommentTok{\# strikes at +/{-} 25 deltas}
\NormalTok{  alpha }\OtherTok{\textless{}{-}} \SpecialCharTok{{-}}\FunctionTok{qnorm}\NormalTok{(.}\DecValTok{25}\SpecialCharTok{*}\FunctionTok{exp}\NormalTok{(d}\SpecialCharTok{*}\NormalTok{T))}
\NormalTok{  Kp }\OtherTok{\textless{}{-}}\NormalTok{ Spot}\SpecialCharTok{*}\FunctionTok{exp}\NormalTok{(}\SpecialCharTok{{-}}\NormalTok{alpha }\SpecialCharTok{*}\NormalTok{ sigma }\SpecialCharTok{*} \FunctionTok{sqrt}\NormalTok{(T)}\SpecialCharTok{+}\NormalTok{(r}\SpecialCharTok{{-}}\NormalTok{d}\SpecialCharTok{+}\NormalTok{(}\DecValTok{1}\SpecialCharTok{/}\DecValTok{2}\NormalTok{)}\SpecialCharTok{*}\NormalTok{sigma}\SpecialCharTok{\^{}}\DecValTok{2}\NormalTok{)}\SpecialCharTok{*}\NormalTok{T)}
\NormalTok{  Kc }\OtherTok{\textless{}{-}}\NormalTok{ Spot}\SpecialCharTok{*}\FunctionTok{exp}\NormalTok{(alpha }\SpecialCharTok{*}\NormalTok{ sigma }\SpecialCharTok{*} \FunctionTok{sqrt}\NormalTok{(T)}\SpecialCharTok{+}\NormalTok{(r}\SpecialCharTok{{-}}\NormalTok{d}\SpecialCharTok{+}\NormalTok{(}\DecValTok{1}\SpecialCharTok{/}\DecValTok{2}\NormalTok{)}\SpecialCharTok{*}\NormalTok{sigma}\SpecialCharTok{\^{}}\DecValTok{2}\NormalTok{)}\SpecialCharTok{*}\NormalTok{T)}
\end{Highlighting}
\end{Shaded}

Define a wrapper function to facilitate calculations on the binary
option:

\begin{Shaded}
\begin{Highlighting}[]
\NormalTok{  O }\OtherTok{\textless{}{-}} \ControlFlowTok{function}\NormalTok{(}\AttributeTok{vol=}\NormalTok{sigma, }\AttributeTok{spot=}\NormalTok{Spot) }\FunctionTok{CashOrNothingOption}\NormalTok{(}\AttributeTok{TypeFlag=}\StringTok{\textquotesingle{}c\textquotesingle{}}\NormalTok{, }\AttributeTok{S=}\NormalTok{spot,}
       \AttributeTok{X=}\NormalTok{X, }\AttributeTok{K=}\DecValTok{100}\NormalTok{, }\AttributeTok{Time=}\NormalTok{T, }\AttributeTok{r=}\NormalTok{r, }\AttributeTok{b=}\NormalTok{b, }\AttributeTok{sigma=}\NormalTok{vol)}\SpecialCharTok{@}\NormalTok{price}
\end{Highlighting}
\end{Shaded}

The Black-Scholes value, using ATM volatility is:

\begin{Shaded}
\begin{Highlighting}[]
  \CommentTok{\# Theoretical BS value}
\NormalTok{  TV.BS }\OtherTok{\textless{}{-}} \FunctionTok{O}\NormalTok{()}
  \FunctionTok{print}\NormalTok{(}\FunctionTok{paste}\NormalTok{(}\StringTok{\textquotesingle{}BS value:\textquotesingle{}}\NormalTok{, }\FunctionTok{round}\NormalTok{(TV.BS,}\DecValTok{2}\NormalTok{)))}
\end{Highlighting}
\end{Shaded}

\begin{verbatim}
## [1] "BS value: 31.46"
\end{verbatim}

For comparison, we can approximate the binary option with a call spread,
giving a value of:

\begin{Shaded}
\begin{Highlighting}[]
  \CommentTok{\# Replication value with call spread}
\NormalTok{  N }\OtherTok{\textless{}{-}} \DecValTok{1000}
\NormalTok{  TV.CS }\OtherTok{\textless{}{-}}\NormalTok{ N}\SpecialCharTok{*}\NormalTok{(}\FunctionTok{GBSOption}\NormalTok{(}\StringTok{\textquotesingle{}c\textquotesingle{}}\NormalTok{, Spot, X}\DecValTok{{-}100}\SpecialCharTok{/}\NormalTok{(}\DecValTok{2}\SpecialCharTok{*}\NormalTok{N), T, r, b, sigma}\SpecialCharTok{+}\FunctionTok{smile}\NormalTok{(X}\DecValTok{{-}100}\SpecialCharTok{/}\NormalTok{(}\DecValTok{2}\SpecialCharTok{*}\NormalTok{N)))}\SpecialCharTok{@}\NormalTok{price }\SpecialCharTok{{-}}
              \FunctionTok{GBSOption}\NormalTok{(}\StringTok{\textquotesingle{}c\textquotesingle{}}\NormalTok{, Spot, X}\SpecialCharTok{+}\DecValTok{100}\SpecialCharTok{/}\NormalTok{(}\DecValTok{2}\SpecialCharTok{*}\NormalTok{N), T, r, b, sigma}\SpecialCharTok{+}\FunctionTok{smile}\NormalTok{(X}\SpecialCharTok{+}\DecValTok{100}\SpecialCharTok{/}\NormalTok{(}\DecValTok{2}\SpecialCharTok{*}\NormalTok{N)))}\SpecialCharTok{@}\NormalTok{price)}

  \FunctionTok{print}\NormalTok{(}\FunctionTok{paste}\NormalTok{(}\StringTok{\textquotesingle{}Value, approximated by a call spread:\textquotesingle{}}\NormalTok{, }\FunctionTok{round}\NormalTok{(TV.BS,}\DecValTok{2}\NormalTok{)))}
\end{Highlighting}
\end{Shaded}

\begin{verbatim}
## [1] "Value, approximated by a call spread: 31.46"
\end{verbatim}

We next define the benchmark instruments:

\begin{Shaded}
\begin{Highlighting}[]
  \CommentTok{\# Put}
\NormalTok{  P }\OtherTok{\textless{}{-}} \ControlFlowTok{function}\NormalTok{(}\AttributeTok{vol=}\NormalTok{sigma, }\AttributeTok{spot=}\NormalTok{Spot) }\FunctionTok{GBSOption}\NormalTok{(}\AttributeTok{TypeFlag=}\StringTok{\textquotesingle{}p\textquotesingle{}}\NormalTok{, }\AttributeTok{S=}\NormalTok{spot, }\AttributeTok{X=}\NormalTok{Kp,}
                             \AttributeTok{Time=}\NormalTok{T, }\AttributeTok{r=}\NormalTok{r, }\AttributeTok{b=}\NormalTok{b, }\AttributeTok{sigma=}\NormalTok{vol)}\SpecialCharTok{@}\NormalTok{price}

  \CommentTok{\# Call}
\NormalTok{  C }\OtherTok{\textless{}{-}} \ControlFlowTok{function}\NormalTok{(}\AttributeTok{vol=}\NormalTok{sigma, }\AttributeTok{spot=}\NormalTok{Spot) }\FunctionTok{GBSOption}\NormalTok{(}\AttributeTok{TypeFlag=}\StringTok{\textquotesingle{}c\textquotesingle{}}\NormalTok{, }\AttributeTok{S=}\NormalTok{spot, }\AttributeTok{X=}\NormalTok{Kc,}
                             \AttributeTok{Time=}\NormalTok{T, }\AttributeTok{r=}\NormalTok{r, }\AttributeTok{b=}\NormalTok{b, }\AttributeTok{sigma=}\NormalTok{vol)}\SpecialCharTok{@}\NormalTok{price}

  \CommentTok{\# Straddle}
\NormalTok{  S }\OtherTok{\textless{}{-}} \ControlFlowTok{function}\NormalTok{(}\AttributeTok{vol=}\NormalTok{sigma, }\AttributeTok{spot=}\NormalTok{Spot) \{}
    \FunctionTok{GBSOption}\NormalTok{(}\AttributeTok{TypeFlag=}\StringTok{\textquotesingle{}c\textquotesingle{}}\NormalTok{, }\AttributeTok{S=}\NormalTok{spot, }\AttributeTok{X=}\NormalTok{Spot, }\AttributeTok{Time=}\NormalTok{T, }\AttributeTok{r=}\NormalTok{r, }\AttributeTok{b=}\NormalTok{b, }\AttributeTok{sigma=}\NormalTok{vol)}\SpecialCharTok{@}\NormalTok{price }\SpecialCharTok{+}
    \FunctionTok{GBSOption}\NormalTok{(}\AttributeTok{TypeFlag=}\StringTok{\textquotesingle{}p\textquotesingle{}}\NormalTok{, }\AttributeTok{S=}\NormalTok{spot, }\AttributeTok{X=}\NormalTok{Spot, }\AttributeTok{Time=}\NormalTok{T, }\AttributeTok{r=}\NormalTok{r, }\AttributeTok{b=}\NormalTok{b, }\AttributeTok{sigma=}\NormalTok{vol)}\SpecialCharTok{@}\NormalTok{price}
\NormalTok{  \}}

  \CommentTok{\# Risk Reversal}
\NormalTok{  RR }\OtherTok{\textless{}{-}} \ControlFlowTok{function}\NormalTok{(vol, }\AttributeTok{spot=}\NormalTok{Spot) \{}
    \FunctionTok{P}\NormalTok{(vol, spot)}\SpecialCharTok{{-}}\FunctionTok{C}\NormalTok{(vol, spot)}
\NormalTok{  \}}

  \CommentTok{\# Butterfly}
\NormalTok{  BF }\OtherTok{\textless{}{-}} \ControlFlowTok{function}\NormalTok{(vol, }\AttributeTok{spot=}\NormalTok{Spot, }\AttributeTok{beta=}\DecValTok{1}\NormalTok{) \{}
\NormalTok{    beta}\SpecialCharTok{*}\NormalTok{(}\FunctionTok{P}\NormalTok{(vol, spot)}\SpecialCharTok{+}\FunctionTok{C}\NormalTok{(vol, spot))}\SpecialCharTok{{-}}\FunctionTok{S}\NormalTok{(vol,spot)}
\NormalTok{  \}}
\end{Highlighting}
\end{Shaded}

The butterfly must be vega-neutral. This is obtained by solving for
\(\beta\):

\begin{Shaded}
\begin{Highlighting}[]
\NormalTok{  BF.V }\OtherTok{\textless{}{-}} \ControlFlowTok{function}\NormalTok{(vol, beta) \{}
\NormalTok{    (}\FunctionTok{BF}\NormalTok{(vol}\SpecialCharTok{+}\NormalTok{eps, }\AttributeTok{beta=}\NormalTok{beta)}\SpecialCharTok{{-}}\FunctionTok{BF}\NormalTok{(vol}\SpecialCharTok{{-}}\NormalTok{eps, }\AttributeTok{beta=}\NormalTok{beta))}\SpecialCharTok{/}\NormalTok{(}\DecValTok{2}\SpecialCharTok{*}\NormalTok{eps)}
\NormalTok{  \}}

\NormalTok{  beta }\OtherTok{\textless{}{-}} \FunctionTok{uniroot}\NormalTok{(}\ControlFlowTok{function}\NormalTok{(b) }\FunctionTok{BF.V}\NormalTok{(sigma, b), }\FunctionTok{c}\NormalTok{(}\DecValTok{1}\NormalTok{, }\FloatTok{1.5}\NormalTok{))}\SpecialCharTok{$}\NormalTok{root}
\end{Highlighting}
\end{Shaded}

Next, we compute the risk indicators for the binary option:

\begin{Shaded}
\begin{Highlighting}[]
\NormalTok{  O.vega }\OtherTok{\textless{}{-}} \FunctionTok{Vega}\NormalTok{(O, sigma)}
\NormalTok{  O.vanna }\OtherTok{\textless{}{-}} \FunctionTok{Vanna}\NormalTok{(O, sigma)}
\NormalTok{  O.volga }\OtherTok{\textless{}{-}} \FunctionTok{Volga}\NormalTok{(O, sigma)}
\end{Highlighting}
\end{Shaded}

and for the benchmark instruments:

\begin{Shaded}
\begin{Highlighting}[]
\NormalTok{  S.vega }\OtherTok{\textless{}{-}} \FunctionTok{Vega}\NormalTok{(S, sigma)}
\NormalTok{  S.vanna }\OtherTok{\textless{}{-}} \FunctionTok{Vanna}\NormalTok{(S, sigma)}
\NormalTok{  S.volga }\OtherTok{\textless{}{-}} \FunctionTok{Volga}\NormalTok{(S, sigma)}

\NormalTok{  RR.vega }\OtherTok{\textless{}{-}} \FunctionTok{Vega}\NormalTok{(RR, sigma)}
\NormalTok{  RR.vanna }\OtherTok{\textless{}{-}} \FunctionTok{Vanna}\NormalTok{(RR, sigma)}
\NormalTok{  RR.volga }\OtherTok{\textless{}{-}} \FunctionTok{Volga}\NormalTok{(RR, sigma)}

\NormalTok{  BF.vega }\OtherTok{\textless{}{-}} \DecValTok{0}
\NormalTok{  BF.vanna }\OtherTok{\textless{}{-}} \FunctionTok{Vanna}\NormalTok{(BF, sigma)}
\NormalTok{  BF.volga }\OtherTok{\textless{}{-}} \FunctionTok{Volga}\NormalTok{(BF, sigma)}
\end{Highlighting}
\end{Shaded}

By definition the smile cost of the straddle is zero, since it is priced
with ATM volatility. For the other two benchmark instruments, the smile
cost is the difference between the price with the smile effect and the
price at the ATM volatility:

\begin{Shaded}
\begin{Highlighting}[]
  \CommentTok{\# RR and BF cost}
\NormalTok{  RR.cost }\OtherTok{\textless{}{-}}\NormalTok{ (}\FunctionTok{P}\NormalTok{(sigma}\SpecialCharTok{+}\FunctionTok{smile}\NormalTok{(Kp))}\SpecialCharTok{{-}}\FunctionTok{C}\NormalTok{(sigma}\SpecialCharTok{+}\FunctionTok{smile}\NormalTok{(Kc)))}\SpecialCharTok{{-}}\NormalTok{(}\FunctionTok{P}\NormalTok{(sigma)}\SpecialCharTok{{-}}\FunctionTok{C}\NormalTok{(sigma))}
\NormalTok{  BF.cost }\OtherTok{\textless{}{-}}\NormalTok{ beta}\SpecialCharTok{*}\NormalTok{(}\FunctionTok{P}\NormalTok{(sigma}\SpecialCharTok{+}\FunctionTok{smile}\NormalTok{(Kp))}\SpecialCharTok{+}\FunctionTok{C}\NormalTok{(sigma}\SpecialCharTok{+}\FunctionTok{smile}\NormalTok{(Kc)))}\SpecialCharTok{{-}}\NormalTok{ beta}\SpecialCharTok{*}\NormalTok{(}\FunctionTok{P}\NormalTok{(sigma)}\SpecialCharTok{+}\FunctionTok{C}\NormalTok{(sigma))}
\end{Highlighting}
\end{Shaded}

We can now compute the price correction for the binary option. First the
approximate method, ignoring the off-diagonal terms in matrix \(A\):

\begin{Shaded}
\begin{Highlighting}[]
  \CommentTok{\# approximate method}
\NormalTok{  CA }\OtherTok{\textless{}{-}}\NormalTok{ RR.cost }\SpecialCharTok{*}\NormalTok{ (O.vanna}\SpecialCharTok{/}\NormalTok{RR.vanna) }\SpecialCharTok{+}\NormalTok{ BF.cost}\SpecialCharTok{*}\NormalTok{(O.volga}\SpecialCharTok{/}\NormalTok{BF.volga)}
\end{Highlighting}
\end{Shaded}

then the more accurate method, solving the \(3\times3\) linear system:

\begin{Shaded}
\begin{Highlighting}[]
  \CommentTok{\# full calculation}
\NormalTok{  A }\OtherTok{\textless{}{-}} \FunctionTok{matrix}\NormalTok{(}\FunctionTok{c}\NormalTok{(S.vega, S.vanna, S.volga,}
\NormalTok{                RR.vega, RR.vanna, RR.volga,}
\NormalTok{                BF.vega, BF.vanna, BF.volga), }\AttributeTok{nrow=}\DecValTok{3}\NormalTok{)}

\NormalTok{  x }\OtherTok{\textless{}{-}} \FunctionTok{matrix}\NormalTok{(}\FunctionTok{c}\NormalTok{(O.vega, O.vanna, O.volga), }\AttributeTok{nrow=}\DecValTok{3}\NormalTok{)}
\NormalTok{  w }\OtherTok{\textless{}{-}} \FunctionTok{solve}\NormalTok{(A, x)}
\NormalTok{  CF }\OtherTok{\textless{}{-}} \FunctionTok{t}\NormalTok{(w) }\SpecialCharTok{\%*\%} \FunctionTok{matrix}\NormalTok{(}\FunctionTok{c}\NormalTok{(}\DecValTok{0}\NormalTok{, RR.cost, BF.cost), }\AttributeTok{nrow=}\DecValTok{3}\NormalTok{)}
\end{Highlighting}
\end{Shaded}

In summary, we get:

\begin{itemize}
\item
  Black-Scholes price: \(31.46\)
\item
  With approximate Vanna-Volga correction:
  \(31.46 + (-4.95) = 26.51\)
\item
  With acurate Vanna-Volga correction:
  \(31.46 + (-3.5) = 27.96\)
\item
  the approximation by a call spread is: \(28.79\)
\end{itemize}

It is worth noting that a naive calculation, where one would plug the
ATM volatility plus smile into the binary option pricing model would
yield a very inaccurate result:

\begin{Shaded}
\begin{Highlighting}[]
\NormalTok{  P.smile }\OtherTok{\textless{}{-}} \FunctionTok{O}\NormalTok{(}\AttributeTok{vol=}\NormalTok{sigma}\SpecialCharTok{+}\FunctionTok{smile}\NormalTok{(X))}
\end{Highlighting}
\end{Shaded}

which yields a value of \(31.54\). Figure~\ref{fig:vv-6}
compares the values of binary options for a range of strikes, computed
with four methods. :

\begin{figure}
\includegraphics{307-BS-VannaVolga_files/figure-latex/vv-6-1} \caption[Price of a digital call in the Black-Scholes framework]{Price of a digital call in the Black-Scholes framework: (1) vanilla Black-Scholes (2) Diagonal VV adjustment (3) Full VV adjustment (4)  Approximation by a call spread}\label{fig:vv-6}
\end{figure}


\printindex

\end{document}
